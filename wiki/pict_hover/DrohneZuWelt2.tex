\documentclass[border=0.5cm,varwidth=\maxdimen]{standalone}
\usepackage[fleqn]{amsmath}

\begin{document}
	Wir transformieren den aktuellen Geschwindigkeitsvektor ${v}_{Drohne}$ vom drohnenfesten Koordinatensystem\\
	in das geodätische Koordinatensystem:
	\begin{flalign*}
		{v}_{d;Drohne} = \begin{pmatrix}
		{v}_{d;Drohne;x} \\
		{v}_{d;Drohne;y} \\
		{v}_{d;Drohne;z}
		\end{pmatrix}
	\end{flalign*} \\
	Daraus entsteht folgede Gleichung: \\
	\begin{flalign*}
		{v}_{g;Drohne} &= {M}_{gd} \cdot {v}_{d;Drohne} \\
		&=\begin{pmatrix}
			cos(\theta)cos(\psi) & sin(\phi)sin(\theta)cos(\psi)-cos(\phi)sin(\psi) & cos(\phi)sin(\theta)cos(\psi)+sin(\phi)sin(\psi) \\
			cos(\theta)sin(\psi) & sin(\phi)sin(\theta)sin(\psi)+cos(\phi)sin(\psi) & cos(\phi)sin(\theta)sin(\psi)-sin(\phi)cos(\psi) \\
			-sin(\theta) & sin(\phi)cos(\theta) & cos(\phi)cos(\theta) \\
		\end{pmatrix}
		\cdot 
		\begin{pmatrix}
		{v}_{d;Drohne;x} \\
		{v}_{d;Drohne;y} \\
		{v}_{d;Drohne;z}
		\end{pmatrix}
	\end{flalign*} \\
\end{document}