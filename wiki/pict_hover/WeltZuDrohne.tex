\documentclass[border=0.5cm,varwidth=\maxdimen]{standalone}
\usepackage[fleqn]{amsmath}

\begin{document}
	Drehung mit dem Winkel $\psi$ um eine z-Achse: \\
	\begin{flalign*}
		R_{z}(\psi) = 
		\begin{pmatrix}
			cos(\psi) & sin(\psi) & 0\\
			-sin(\psi) & cos(\psi) & 0\\
			0 & 0 & 1
		\end{pmatrix}
	\end{flalign*}\\
	Drehung mit dem Winkel $\theta$ um eine y-Achse: \\
	\begin{flalign*}
		R_{y}(\theta) = 
		\begin{pmatrix}
			cos(\theta) & 0 & -sin(\theta)\\
			0 & 1 & 0\\
			sin(\theta) & 0 & cos(\theta)
		\end{pmatrix}
	\end{flalign*}\\
	Drehung mit dem Winkel $\phi$ um eine x-Achse: \\
	\begin{flalign*}
		R_{x}(\phi) =
		\begin{pmatrix}
			1 & 0 & 0\\
			0 & cos(\phi) & sin(\phi)\\
			0 & -sin(\phi) & cos(\phi)
		\end{pmatrix}
	\end{flalign*}\\
	Gesamttransformationsmatrix bei einer Drehreihenfolge $\psi \rightarrow \theta \rightarrow \phi$ ("von rechts lesen") - \\entspricht einer Transformation vom geodätischen ins drohnenfeste Koordinatensystem:\\
	\begin{flalign*}
		{M}_{dg} &= R_{z}(\psi) \cdot R_{y}(\theta) \cdot R_{x}(\phi)\\
		&= 
		\begin{pmatrix}
			1 & 0 & 0\\
			0 & cos(\phi) & sin(\phi)\\
			0 & -sin(\phi) & cos(\phi)
		\end{pmatrix}
		\cdot
		\begin{pmatrix}
			cos(\theta) & 0 & -sin(\theta)\\
			0 & 1 & 0\\
			sin(\theta) & 0 & cos(\theta)
		\end{pmatrix}
		\cdot
		\begin{pmatrix}
			cos(\psi) & sin(\psi) & 0\\
			-sin(\psi) & cos(\psi) & 0\\
			0 & 0 & 1
		\end{pmatrix}\\
		&=\begin{pmatrix}
			cos(\theta)cos(\psi) & cos(\theta)sin(\psi) & -sin(\theta) \\
			sin(\phi)sin(\theta)cos(\psi)-cos(\phi)sin(\psi) & sin(\phi)sin(\theta)sin(\psi)+cos(\phi)cos(\psi) & sin(\phi)cos(\theta) \\
			cos(\phi)sin(\theta)cos(\psi)+sin(\phi)sin(\psi) & cos(\phi)sin(\theta)sin(\psi)-sin(\phi)cos(\psi) & cos(\phi)cos(\theta) \\
		\end{pmatrix}
	\end{flalign*}
\end{document}